\documentclass[aspectratio=169]{beamer}
\usepackage[utf8]{inputenc}
\usepackage[spanish]{babel}
\usepackage{amsmath}
\usepackage{amssymb}
\usepackage{graphicx}
\usepackage{booktabs}
\usepackage{multirow}
\usepackage{xcolor}

% Tema y colores
\usetheme{Madrid}
\usecolortheme{default}
\setbeamertemplate{navigation symbols}{}
\setbeamertemplate{caption}[numbered]

% Definir colores personalizados
\definecolor{udea}{RGB}{0,102,51}
\setbeamercolor{structure}{fg=udea}
\setbeamercolor{title}{fg=white,bg=udea}
\setbeamercolor{frametitle}{fg=white,bg=udea}
\setbeamercolor{titlelike}{fg=white,bg=udea}
\setbeamercolor{author}{fg=black}
\setbeamercolor{institute}{fg=black}
\setbeamercolor{date}{fg=black}

% Información del documento
\title[Modelo de Ising - Monte Carlo]{Simulación Monte Carlo del Modelo de Ising en Diferentes Tipos de Redes}
\author[]{Juan Montoya \& Melanny Silva \& Pablo Sánchez}
\institute[UdeA]{Instituto de Física\\Universidad de Antioquia}
\date{\today}

\begin{document}

%===============================================================================
% DIAPOSITIVA 1: PORTADA
%===============================================================================
\begin{frame}
\titlepage
\end{frame}

%===============================================================================
% DIAPOSITIVA 2: CONTENIDO
%===============================================================================
\begin{frame}{Contenido}
\tableofcontents
\end{frame}

%===============================================================================
% DIAPOSITIVA 3: INTRODUCCIÓN
%===============================================================================
\section{Introducción}
\begin{frame}{Introducción}
\begin{block}{Modelo de Ising}
Sistema magnético con espines $S_i = \pm 1$ que pueden orientarse en dos direcciones opuestas
\end{block}

\begin{columns}[T]
\column{0.5\textwidth}
\textbf{Importancia:}
\begin{itemize}
\item Transiciones de fase
\item Magnetización espontánea
\end{itemize}

\column{0.5\textwidth}
\textbf{Soluciones analíticas:}
\begin{itemize}
\item 1D: No hay transición (Ising, 1925)
\item 2D cuadrada: (Onsager, 1944)
\item Otros casos: Métodos numéricos
\end{itemize}
\end{columns}

\vspace{0.1cm}
\begin{alertblock}{Objetivo}
Estudiar el modelo de Ising en diferentes topologías de red (z=2,3,4,8) con dilución magnética usando Monte Carlo
\end{alertblock}
\end{frame}

%===============================================================================
% DIAPOSITIVA 4: MARCO TEÓRICO - HAMILTONIANO
%===============================================================================
\section{Marco Teórico}
\begin{frame}{Marco Teórico: Hamiltoniano}
\begin{block}{Hamiltoniano del Modelo de Ising}
$$H = -J \sum_{\langle i,j \rangle} S_i S_j - H \sum_i S_i$$
\end{block}

\begin{columns}[T]
\column{0.5\textwidth}
\textbf{Parámetros:}
\begin{itemize}
\item $S_i = \pm 1$: espín en sitio $i$
\item $J$: constante de acoplamiento
\item $H$: campo magnético externo
\item $\langle i,j \rangle$: primeros vecinos
\end{itemize}

\column{0.5\textwidth}
\textbf{Casos de estudio:}
\begin{itemize}
\item \textcolor{blue}{Paramagneto}: $J=0$
\item \textcolor{red}{Ferromagneto}: $J>0$
\end{itemize}

\vspace{0.01cm}
\textbf{Magnetización:}
$$m = \frac{1}{n} \sum_i S_i$$
\end{columns}

\vspace{0.05cm}
\begin{exampleblock}{Dilución magnética}
Factor $q = n/N$: fracción de sitios ocupados ($q = 0, 0\text{.}5, 0\text{.}8$)
\end{exampleblock}
\end{frame}

%===============================================================================
% DIAPOSITIVA 5: MÉTODO MONTE CARLO
%===============================================================================
\begin{frame}{Método Monte Carlo - Algoritmo de Metropolis}
\begin{block}{Distribución de Boltzmann}
$$P(\mu) \propto e^{-\beta E_\mu}, \quad \beta = \frac{1}{k_B T}$$
\end{block}

\begin{columns}[T]
\column{0.5\textwidth}
\textbf{Algoritmo de Metropolis:}
\begin{enumerate}
\item Seleccionar espín aleatorio
\item Calcular $\Delta E$ al invertir espín
\item Si $\Delta E \leq 0$: aceptar
\item Si $\Delta E > 0$: aceptar con
$P = e^{-\beta \Delta E}$
\item Repetir hasta equilibrio
\end{enumerate}

\column{0.5\textwidth}
\textbf{Parámetros de simulación:}
\begin{itemize}
\item $L$: tamaño de red
\item Pasos MC: 4000-10000
\item Condiciones periódicas
\item Optimización: Numba JIT
\end{itemize}
\end{columns}

\vspace{0.01cm}
\begin{alertblock}{Ley de estados correspondientes (Paramagneto)}
$$m = \tanh\left(\frac{H}{k_B T}\right)$$
\end{alertblock}
\end{frame}

%===============================================================================
% DIAPOSITIVA 6B: VISUALIZACIÓN DE TOPOLOGÍAS
%===============================================================================
\begin{frame}{Visualización de las Topologías de Red}
\begin{center}
\includegraphics[width=0.6\textwidth]{topo.png}
\end{center}

\begin{columns}[T]
\column{0.5\textwidth}
\textbf{Estructuras 1D y 2D:}
\begin{itemize}
\item \textbf{(A) Cadena 1D}: $z=2$ vecinos
\item \textbf{(B) Red Cuadrada}: $z=4$ vecinos
\item \textbf{(C) Red Hexagonal}: $z=3$ vecinos
\end{itemize}

\column{0.5\textwidth}
\textbf{Estructura 3D:}
\begin{itemize}
\item \textbf{(D) Red BCC}: $z=8$ vecinos
\end{itemize}

\vspace{0.3cm}
\textbf{Código de colores:}
\begin{itemize}
\item \textcolor{orange}{Naranja}: sitio central
\item \textcolor{yellow}{Amarillo}: vecinos
\item Azul: otros sitios
\end{itemize}
\end{columns}

\end{frame}

%===============================================================================
% DIAPOSITIVA 6: TOPOLOGÍAS DE RED
%===============================================================================
\section{Implementación y Topologías}
\begin{frame}{Topologías de Red Estudiadas}
\begin{center}
\begin{tabular}{cccc}
\toprule
\textbf{Topología} & \textbf{Dimensión} & \textbf{z} & \textbf{Tamaño (L)} \\
\midrule
Cadena 1D & 1D & 2 & 100 \\
Hexagonal (Honeycomb) & 2D & 3 & 30 \\
Cuadrada & 2D & 4 & 30 \\
BCC (Cúbica Centrada) & 3D & 8 & 15 \\
\bottomrule
\end{tabular}
\end{center}
\begin{center}
    \textbf{Características:}
    \begin{itemize}
    \item Condiciones de frontera periódicas
    \item $z$ = número de coordinación
    \item Impacto en $T_c$
    \item Dilución: $q \in \{0, 0\text{.}5, 0\text{.}8\}$
    \end{itemize}
\end{center}
\end{frame}

%===============================================================================
% DIAPOSITIVA 7: RESULTADOS PARAMAGNETISMO - CURVAS m vs H
%===============================================================================
\section{Resultados: Paramagnetismo}
\begin{frame}{Paramagnetismo (J=0): Curvas m vs. H}
\begin{columns}[T]
\column{0.6\textwidth}
\vspace{-0.8cm}
\begin{center}
\includegraphics[width = 0.75\textwidth]{mH.png}
\end{center}

\column{0.4\textwidth}
\textbf{Observaciones:}
\begin{itemize}
\item Curvas para $T_1 < T_2 < T_3$
\item Mayor T $\Rightarrow$ respuesta más lineal
\end{itemize}

\vspace{0.3cm}
\textbf{Validación cuantitativa:}
\begin{itemize}
\item $R^2 > 0.99999$ con teoría
\end{itemize}
\end{columns}

\end{frame}

%===============================================================================
% DIAPOSITIVA 8: LEY DE ESTADOS CORRESPONDIENTES
%===============================================================================
\begin{frame}{Ley de Estados Correspondientes}
\begin{columns}[T]
\column{0.6\textwidth}
\vspace{-0.75cm}
\begin{center}
\includegraphics[width =\textwidth]{mHT.png}
\end{center}

\column{0.4\textwidth}
\textbf{Verificación:}
\begin{itemize}
\item Todas las curvas se ajustan
\item $m = \tanh(H/T)$
\item Válido para todas las $T$
\item Independiente de topología
\end{itemize}

\vspace{0.3cm}
\textbf{Métricas:}
\begin{itemize}
\item $R^2$: 0.999995
\end{itemize}
\end{columns}
\end{frame}

%===============================================================================
% DIAPOSITIVA 13: EFECTO DEL NÚMERO DE COORDINACIÓN
%===============================================================================
\section{Análisis y Discusión}
\begin{frame}{Efecto del Número de Coordinación (z)}
\begin{columns}[T]
\column{0.7\textwidth}
\vspace{-0.6cm}
\begin{center}
\includegraphics[width = \textwidth]{zq.png}
\end{center}

\column{0.3\textwidth}

\end{columns}
\end{frame}


%===============================================================================
% DIAPOSITIVA 9: RELAJACIÓN ENERGÉTICA
%===============================================================================
\begin{frame}{Relajación del Sistema: Energía vs. Pasos MC}
\begin{columns}[T]
\column{0.6\textwidth}
\vspace{-0.8cm}
\begin{center}
\includegraphics[width = \textwidth]{UMC.png}
\end{center}

\column{0.4\textwidth}
\textbf{Análisis de equilibración:}
\begin{itemize}
\item Estado inicial: aleatorio
\item Fluctuaciones térmicas
\item Convergente
\end{itemize}

\vspace{0.3cm}
\textbf{Comportamiento:}
\begin{itemize}
\item Mayor $z$ $\Rightarrow$ $|E|$ mayor
\end{itemize}
\end{columns}

\end{frame}

%===============================================================================
% DIAPOSITIVA 9: ERGONOCIDAD
%===============================================================================
\begin{frame}{Relajación del Sistema: Energía vs. Pasos MC - Ergonocidad}
\begin{columns}[T]
\column{0.6\textwidth}
\vspace{-0.8cm}
\begin{center}
\includegraphics[width = \textwidth]{Ergo.png}
\end{center}

\column{0.4\textwidth}
\textbf{Análisis de equilibración:}
\begin{itemize}
\item Estado inicial: aleatorio
\item Fluctuaciones térmicas
\item Convergente
\end{itemize}

\vspace{0.3cm}
\textbf{Comportamiento:}
\begin{itemize}
\item Mayor $z$ $\Rightarrow$ $|E|$ mayor
\end{itemize}
\end{columns}

\end{frame}


%===============================================================================
% DIAPOSITIVA 10: RESULTADOS FERROMAGNETISMO - HISTÉRESIS
%===============================================================================
\section{Resultados: Ferromagnetismo}
\begin{frame}{Ferromagnetismo (J=1): Ciclos de Histéresis}

\begin{center}
\includegraphics[width = 0.65\textwidth]{Hyz=3.png}
\end{center}

\end{frame}

\section{Resultados: Ferromagnetismo}
\begin{frame}{Ferromagnetismo (J=1): Ciclos de Histéresis}
\begin{center}
\includegraphics[width = 0.65\textwidth]{Hyz=4.png}
\end{center}



\end{frame}

%===============================================================================
% DIAPOSITIVA 11: MAGNETIZACIÓN VS TEMPERATURA
%===============================================================================
\begin{frame}{Magnetización vs. Temperatura: Transición de Fase}
\begin{columns}[T]
\column{0.65\textwidth}
\vspace{-0.8cm}
\begin{center}
\includegraphics[width = 1.01\textwidth]{TC.png}
\end{center}

\column{0.35\textwidth}
\textbf{Observacion:}
\begin{itemize}
\item $T_c$ depende de $z$ y $q$
\end{itemize}

\vspace{0.3cm}
\textbf{Temperaturas críticas:}
\begin{itemize}
\item $z=2$ (1D): No transición
\item $z=3$: $T_c \approx 1.0$
\item $z=4$: $T_c \approx 2.27$
\item $z=8$: $T_c \approx 6.3$
\end{itemize}
\end{columns}

\end{frame}

%===============================================================================
% DIAPOSITIVA 13: EFECTO DEL NÚMERO DE COORDINACIÓN
%===============================================================================
\section{Análisis y Discusión}
\begin{frame}{Efecto del Número de Coordinación (z)}
\begin{columns}[T]
\column{0.7\textwidth}
\vspace{-0.6cm}
\begin{center}
\includegraphics[width = \textwidth]{zqJ.png}
\end{center}

\column{0.3\textwidth}
\textbf{Resultados ($q=0.8$):}
\begin{itemize}
\item $z=2$: Sin transición
\item $z=3$: $T_c \sim 1.0$
\item $z=4$: $T_c \sim 2.27$
\item $z=8$: $T_c \sim 6.3$
\end{itemize}

\end{columns}
\end{frame}

%===============================================================================
% DIAPOSITIVA 12: MICROESTADOS MAGNÉTICOS
%===============================================================================
\begin{frame}{Microestados Magnéticos (Snapshots)}
\begin{columns}[T]
\column{0.55\textwidth}
\vspace{-0.8cm}
\begin{center}
\includegraphics[width = \textwidth]{Microesta_Cuad2D.png}
\end{center}

\column{0.45\textwidth}
\textbf{Configuraciones:}
\begin{itemize}
\item \textcolor{red}{$\uparrow$}: espín arriba (rojo)
\item \textcolor{blue}{$\downarrow$}: espín abajo (azul)
\item Blanco: sitio desocupado ($q<1$)
\end{itemize}

\vspace{0.3cm}
\textbf{Análisis visual:}
\begin{itemize}
\item $T \ll T_c$: Dominios grandes
\item $T \gg T_c$: Configuración aleatoria
\end{itemize}
\end{columns}

\end{frame}

%===============================================================================
% DIAPOSITIVA 15: CONCLUSIONES
%===============================================================================
\section{Conclusiones}
\begin{frame}{Conclusiones}
\begin{columns}[T]
\column{0.5\textwidth}
\textbf{Paramagnetismo (J=0):}
\begin{itemize}
\item Verificación de $m = \tanh(H/T)$
\item $R^2 > 0.99999$
\item Ley de estados correspondientes confirmada
\item Independiente de $z$ y $q$
\end{itemize}

\vspace{0.3cm}
\textbf{Ferromagnetismo (J=1):}
\begin{itemize}
\item Transiciones de fase de 2º orden
\item $T_c$ aumenta con $z$
\item $T_c$ disminuye con dilución
\item $T_c$(z=4) $\approx$ 2.27 
\end{itemize}

\column{0.5\textwidth}
\textbf{Histéresis:}
\begin{itemize}
\item Desaparecen para $T > T_c$
\item Comportamiento físico correcto
\end{itemize}

\vspace{0.3cm}
\textbf{Implementación:}
\begin{itemize}
\item 4 topologías: z=2,3,4,8
\item Algoritmo de Metropolis
\item Optimización con Numba
\end{itemize}
\end{columns}

\end{frame}


%===============================================================================
% DIAPOSITIVA 10: Entro
%===============================================================================
\section{Extras}
\begin{frame}{Entropía}

\begin{center}
\includegraphics[width = 0.65\textwidth]{S.png}
\end{center}

\end{frame}



%===============================================================================
% DIAPOSITIVA FINAL: PREGUNTAS
%===============================================================================
\begin{frame}[plain]
\begin{center}
\vspace{1cm}
\Large{Gracias por su atención}
\end{center}
\end{frame}

\end{document}
